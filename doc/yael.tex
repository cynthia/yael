\documentclass[a4paper,11pt,notitlepage,final,twoside]{report}

% Options possibles : 10pt, 11pt, 12pt (taille de la fonte)
%                     oneside, twoside (recto simple, recto-verso)
%                     draft, final 
%                     titlepage, notitlepage 

\usepackage[utf8]{inputenc}   % LaTeX, comprends les accents !
\usepackage[hmargin=2.5cm,vmargin=2.5cm]{geometry}         % Definir les marges
\usepackage{graphicx}
\usepackage{xspace}
\usepackage{url}

% \pagestyle{headings}        % Pour mettre des entetes avec les titres
                              % des sections en haut de page

\newcommand{\yael}{\textsc{Yael}}

\setlength{\parindent}{0pt}

\title{~\vspace{9cm} \\ \yael: Getting started} 
\author{Matthijs Douze \and Herv\'e J\'egou}
\date{April 20\textsuperscript{th}, 2010} 


\begin{document}

\maketitle
\thispagestyle{empty}

\vfill

\includegraphics[width=3cm]{./figs/logoinria}


\newpage

\chapter*{Licence}

\thispagestyle{empty}

Copyright @ INRIA 2010. \\
Authors: Matthijs Douze \& Herv\'e J\'egou \\
Contact: matthijs.douze@inria.fr  herve.jegou@inria.fr \\
\medskip

This software is a computer program whose purpose is to provide 
efficient tools for basic yet computationally demanding tasks, 
such as find k-nearest neighbors using exhaustive search 
and kmeans clustering. 
\medskip

This software is governed by the CeCILL license under French law and
abiding by the rules of distribution of free software.  You can  use, 
modify and/ or redistribute the software under the terms of the CeCILL
license as circulated by CEA, CNRS and INRIA at the following URL
\url{http://www.cecill.info}, and provided within the \yael library 
in the file LICENSE. 
\medskip

As a counterpart to the access to the source code and  rights to copy,
modify and redistribute granted by the license, users are provided only
with a limited warranty  and the software's author,  the holder of the
economic rights,  and the successive licensors  have only  limited
liability. 
\medskip

In this respect, the user's attention is drawn to the risks associated
with loading,  using,  modifying and/or developing or reproducing the
software by the user in light of its specific status of free software,
that may mean  that it is complicated to manipulate,  and  that  also
therefore means  that it is reserved for developers  and  experienced
professionals having in-depth computer knowledge. Users are therefore
encouraged to load and test the software's suitability as regards their
requirements in conditions enabling the security of their systems and/or 
data to be ensured and,  more generally, to use and operate it in the 
same conditions as regards security. 
\medskip

The fact that you are presently reading this means that you have had
knowledge of the CeCILL license and that you accept its terms.


\tableofcontents            % Table des matieres


\chapter{What is this?}

This is a library for performing efficient basic operations, 
in particular kmeans and exhaustive nearest neighbor search function.
It offers three interfaces: 
\begin{itemize}
\item C, 
\item python, 
\item matlab.
\end{itemize}

The library has been tested under different architectures, in particular
\begin{itemize}
\item Linux 32 bits: Fedora Core 11
\item Linux 64 bits: Fedora Core 10, Fedora Core 11, Ubuntu Karmic, Debian 4.1.2-25
\end{itemize}

The library has not been packaged nor tested for Microsoft Windows. 



\chapter{Getting Started}

\section{Content}

\section{Prerequisites}

The library requires the following software/libraries to be installed. 
Some of them are related to the non-core interface (python and matlab) 
and are not strictly mandatory. 
\bigskip

{\bf 1-- Blas and Lapack. } 
Any implementation should work as it is
wrapped with the Fortran calling conventions. 
It might be required to adjust the location of these libraries in 
the makefile.inc file generated by the \texttt{configure.sh} script. 
The \texttt{LD\_LIBRARY\_PATH} environment variable should be set accordingly
\bigskip

{\bf 2-- Python-dev}. 
This library is usually obtained the python-dev package. 
Note that having python installed on your machine does not necessarily 
means that the developpement kit is installed as well. 
If only python is installed, not python-dev, you will get an error 
saying that the file \texttt{Python.h} can not be found. 
\bigskip

{\bf 3-- swig.} This package is required to create the python interface. 
\bigskip

{\bf 4-- Matlab.}
\smallskip
This software is used to generate the reference manual (pdf and HTML) from the source 
files. 
\bigskip

{\bf 5-- doxygen.}
\smallskip
This software is used to generate the reference manual (HTML) from the source 
files. 
\bigskip

{\bf 6-- pdflatex.}
\smallskip
This software is used to re-generate this getting started manual. 
\bigskip


These two last pre-requisites are required by the python interface 
and are not strictly mandatory. If you are not interested by this 
python interface and that you don't have python-dev and swig installed, 
you should remove the target \texttt{\_yael.so} target in the Makefile of the 
yael core directory. 


\section{Installation procedure}

1- ./configure.sh in the yael root directory. 
For most configurations, nothing has to be done. However you might need 
to adjust the variables defined in makefile.inc to fit your local config. 

2- make


3- Try to compile the programs in progs and test

At this point, only the core C and python library are compiled. 
If you need the matlab interface, you have to compile the mex files:

4- cd matlab ; make

The mex executable should be in the PATH. 


In order to generate the reference manual and this tutorial, you should go into the 
doc subdirectory and execute the following command line. 

5- cd doc; make


\section{Troubleshooting}



\chapter{C interface and basic programs}

\chapter{Python}

\section{Loading and using \yael}

\section{Acquire pointers}

\section{NumPy interface}


\chapter{Matlab}

\section{Installation}



\appendix                     % Les annexes

\chapter{File exchange format}               % Annexe A

fvecfile and ivecfile



% \listoffigures              % Table des figures

% \listoftables               % Liste des tableaux

\end{document}
